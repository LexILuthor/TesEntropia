\documentclass[ignorenonframetext,slidestop,compress]{beamer}
%\usepackage[active]{srcltx}
\usepackage{mathrsfs}
\usepackage{cancel} % use \cancel{} \cancelto{}{} or \bcancel{} \bcancelto{}{} in math formula
\usepackage{subfigure}

%setta il tema vedi http://mike.polycat.net/gallery/beamer-themes 
%\usetheme{Warsaw}
%\usetheme{Boadilla}
%\usetheme{Malmoe}
%\usetheme{Antibes}
%\usetheme{Bergen}
%\usetheme{Berkeley}
%\usetheme{Berlin}
%\usetheme{boxes}
%\usetheme{Copenhagen}
%\usetheme{Darmstadt}
\usetheme{default}
%\usetheme{Dresden}
%\usetheme{Frankfurt}
%\usetheme{Goettingen}
%\usetheme{Hannover}
%\usetheme{Ilmenau}
%\usetheme{JuanLesPins}
%\usetheme{Luebeck}
%\usetheme{Madrid}
%\usetheme{Malmoe}
%\usetheme{Marburg}
%\usetheme{Montpellier}
%\usetheme{PaloAlto}
%\usetheme{Pittsburgh}
%\usetheme{Rochester}
\usetheme{Singapore}
%\usetheme{Szeged}
%\usetheme{umbc2} 

%setta la combinazione di colori 
%\usecolortheme{albatross}

\setbeamercovered{transparent}

\usepackage[latin1]{inputenc}
\usepackage[italian]{babel}

%%%%%%%%%%%%%%%%%%%%%%%%IMPOSTAZIONI%%%%%%%%%%%%%%%%%%%%%%
%\pgfdeclareimage[height=.5cm]{logo}{logo_uni}
%\pgfdeclareimage[height=1.5cm]{logo_large}{logo_uni}
%\logo{\pgfuseimage{logo}}


\title{\textsc{Action growth for static black holes in modified gravity}}

\author{Sebastiani Lorenzo, Luciano Vanzo, Sergio Zerbini,\\ Phys.Rev. D97 (2018) no.4, 044009 [arXiv:1710.05686 [hep-th]}

\subject{DMC}
\date{}

%%%%%%%%%%%%%%%%%%%%%%%%%%%%%%%%%%%%%%%%%%%%%%%%%%%%%%%%%%%%%%%%%%%


\begin{document}

%1
\frame{\maketitle}

%2
\begin{frame}
{\bf Holographic Complexity and Bulk Action}

\textit{Conjecture [Susskind et al., 2015]}:
In  the  context  of  AdS/CFT  duality, the quantum complexity of a holographic state is dual to the action of a certain space-time region (Wheeler-DeWitt patch).
According with the holographyic principle, the proprieties of a spatial region are encoded in its boundary.
The  conjecture  says 
that the action 
of a certain maximal spacelike slice, which extends into the interior of an AdS black
hole,  is  proportional  to  the  computational  complexity  of  the  instantaneous
boundary  CFT state. Thus, the growth of the complexity is saturated by the Lloyd's bound,
since for the action growth of an AdS black hole it is found
\begin{equation*}
\frac{d I}{dt}=2M\,, 
\end{equation*}
where  $M$ is the energy of the BH. If we modify the interior solution, 
\begin{equation*}
\frac{d I}{dt}\leq 2M\,.
\end{equation*}

\end{frame}


%3
\begin{frame}
%\begin{center}
{\bf Action and boundary terms in $F(R)$-gravity}
%\end{center}
We start from the gravitational action,   
\begin{equation*}
 I=\int_\mathcal{M} d^4 x\sqrt{-g}F(R)\,.\label{action0}
\end{equation*}
A static, spherically simmetric space-time is given by,
\begin{equation*}
 ds^2=-\text{e}^{2\alpha(r)}B(r)+\frac{dr^2}{B(r)}+r^2d\Omega^2_k\,,
\end{equation*}
where $d\Omega^2_k$ is the metric of a constant curvature two-dimensional space-time with spherical, flat or hyperbolic topology,
\begin{equation*}
d\Omega_k^2=\left(\frac{d\rho^2}{1-k\rho^2}+\rho^2 d\phi^2\right)\,,\quad k=0,\pm 1\,.
\end{equation*}
In order to have a well posed variational principle, we need to subtract to the action the boundary terms.

%In order to deal with a standard Lagrangian we can introduce a Lagrangian multiplier $\lambda$ in the following way,
%\begin{eqnarray*}
 % I&=&\int_\mathcal{M} d^4 x\left(\text{e}^{\alpha(r)}r^2\right)\left[F(R)-\lambda\left[R
  %-R(r) \right]\right]\,,
%\end{eqnarray*}
%where $R(r)$ is the explicit form of the Ricci scalar for the given metric, such that the variation with respect to $\lambda$ correctly leads to $R=R(r)$.
\end{frame}

%4
\begin{frame}
After integration by part we get (on shell),
\begin{eqnarray*}
 I &=&V_k\int dt\,\int dr e^{\alpha(r)}\left\{r^2\left(F(R)-F'(R)R\right)
\right.\nonumber\\ &&\left.
 +F'(R)\left(2k+2 r\frac{d B(r)}{dr}+2 B(r)+4r B(r)\frac{d\alpha(r)}
{d r}\right)\right.
\nonumber\\
&& +\left.F''(R)\frac{d R}{d r}r^2\left(\frac{d B(r)}{d r}+2B(r)\frac{d \alpha(r)}{dr}+\frac{4B(r)}{r}\right)\right\}+BT\,,
\end{eqnarray*}
where $V_k$ is the volume of the horizon manifold and the boundary term is given by
\begin{equation*}
\hspace{-0.6cm}BT =
-V_k \int  dt \int dr \frac{d}{dr}\left[
F'(R)\text{e}^{\alpha(r)}r^2\left(\frac{d B(r)}{d r}+2 B(r)\frac{d\alpha(r)}{d r}+\frac{4 B(r)}{r}\right)
\right]\,,
\label{BT}
\end{equation*}
and is a generalization of the on-shell form of the GHY boundary term of GR.
The equations of motion come from the variation 
of the bulk action only.
\end{frame}

%5
\begin{frame}
{\bf Black hole thermodynamics in $F(R)$-gravity}
Given a $F(R)$-black hole solution with horizon $r=r_H$,
\begin{equation*}
B(r_H)=0\,,\quad 0<d B(r)/d r|_H\,, 
\end{equation*}
the First Law of Thermodynamics may be derived from the first equation of motion wich leads to
\begin{equation*}
 T_K d S_W= \left(4\pi\right)V_k\left(2 r_H F'(R_H)dr_H+r_H^2 F''(R_H)\left(\frac{d R}{dr}\right)\Big\vert_H dr_H\right)\,.
\end{equation*}
Here, $T_K=(\text{e}^{\alpha(r_H)}/(4\pi) )d B(r)/dr|_H$ is the Killing temperature associated to the BH-horizon and 
$S_W=(4\pi) F'(R_H)\mathcal A_H$, $\mathcal A_H=V_k r_H^2$, is the Wald entropy. Thus, we obtain an expression for the BH-energy,
\begin{equation*}
E_K:=V_k\int  \text{e}^{\alpha(r_H)}\left(2 k\,F'(R_H)-\left(R_H F'(R_H)-F(R_H)r_H^2\right)\right)d r_H\,.
\end{equation*}
In a large class of $F(R)$-BHs where this formalism holds true, one finds that $E_K$ is proportional to the integration constant of the solution.
\end{frame}


%6
\begin{frame}
{\bf Evaluation of the action growth in $F(R)$-gravity}
The action growth associated with the BH interior ($r<r_H$) is computed in a region where the metric is dynamics.  
If we replace $r$ with a time coordinate $T$ ($0<T<r_H$) and $t$ with a space coordinate $\rho$, we get
\begin{equation*}
 ds^2=-\frac{dT^2}{B(T)}+\text{e}^{2\alpha(T)}B(T)d\rho^2+ T^2d\Omega^2_k\,,\quad  B(r)\rightarrow -B(T)\,.
 \label{imetric}
\end{equation*}
 The action growth can be defined in a covariant way by means
\begin{equation*}
 C= \lim_{T \rightarrow r_H} K^\mu \partial_\mu \hat{I}\,,\quad \hat I\equiv V_k \text{e}^{-\alpha(r_H)}\int_0^{r_H} dT L(T)\,,
\end{equation*} 
where $\hat I$ is the bulk action of the theory and $K^\mu=(0,\text{e}^{-\alpha(T)},0,0)$ is the Kodama vector field. By using the preceding result,
 \begin{eqnarray*}
&& L(T) = e^{\alpha(T)}\left\{T^2\left(F-F'R\right)+2F'\left(k- T\frac{d B(T)}{dT}- B(T)
 \right.\right. \\ &&
\hspace{-1cm}              \left.\left.
 -2T B(T)\frac{d\alpha(T)}
 {d T}\right)
-F''\frac{d R}{d T}T^2\left(\frac{d B(T)}{d T}+2B(T)\frac{d \alpha(T)}{dT}+\frac{4B(T)}{T}\right)\right\}\,.
 \end{eqnarray*}
 \end{frame}

%7 
\begin{frame}
{\bf Action growth: $\alpha(r)=0$ cases.}
This is the constant curvature case $R=4\Lambda$ with AdS metric 
\begin{equation*}
ds^2=-B(r)dt^2+\frac{d r^2}{B(r)}+r^2d\Omega^2_k\,, 
\end{equation*}
 \begin{equation*}
B(r)=k-\frac{c}{r}-\frac{\Lambda r^2}{3}\,,\quad
\Lambda=\frac{R_0 F'(R_0)-F(R_0)}{2 F'(R_0)}<0\,,\label{SdS}
\end{equation*}
where $c$ is a free integration constant. Given a BH solution (i.e. we can find $B(r_H)=0$ with $0<r_H$), the First Law leads to
\begin{equation*}
E_K=2V_k F'(R_H)c\,.
\end{equation*}
The action growth results to be
\begin{equation*}
C=4 V_{k}F'(R)c=2E_K\,,
\end{equation*}
like in General Relativity. 
\end{frame}
 
%8
 \begin{frame}
Let us take the model,
 \begin{equation*}
F(R)=2a k\sqrt{k(R+12\lambda)}\,, 
\end{equation*}
which admits the non-trivial SSS solution with $\alpha(r)=0$, 
\begin{equation*}
ds^2=-B(r)dt^2+\frac{d r^2}{B(r)}+r^2d\Omega^2_k\,, 
\end{equation*}
\begin{equation*}
B(r)=\frac{k}{2}+\frac{c}{r^2}+\lambda r^2\,,\quad R= R=-12\lambda + \frac{k}{r^2}\,,
\end{equation*}
where $c$ is a free constant. The energy of the corresponding black hole is computed as
 \begin{equation*}
 E_K=-3 V_K a c\,,
\end{equation*} 
and for the action growth one finds again  
  \begin{equation*}
 C= -6 V_K a c=2E_K\,.
\end{equation*}
 \end{frame}

%9
\begin{frame}
{\bf Action growth: Clifton-Barrow models}
 Consider the class of Lagrangians, 
\begin{equation*}
F(R)=\frac{R^{\delta+1}}{\kappa}\,,\quad \delta\neq 1\,,
\end{equation*}
with SSS solutions [Clifton \& Barrow (2005)],
\begin{equation*}
ds^2=-\text{e}^{2\alpha(r)}B(r) + \frac{d r^2}{B(r)}+r^2d\Omega^2_k\,, 
\end{equation*}
\begin{equation*}
\text{e}^{2\alpha(r)}=
\left(\frac{r}{r_0}\right)^{2a}
\,,\quad
B(r)=B_0\left(k-\frac{c}{r^b}\right) \,,\quad  R=\frac{R_0}{r^2}\,,
\end{equation*}
where $R_0\,,B_0\,,a$ and $b$ are functions of $\delta$ and $c$ is an integration constant. When the solution describes a black hole, the First Law leads to
\begin{equation*}
E_K=2(1-\delta^2) \frac{V_k}{r_0^a\kappa}R_0^\delta B_0 c\,.
\end{equation*}
\end{frame} 
 
%10
\begin{frame}
Now the action growth results to be
\begin{equation*}
C=
\frac{V_k e^{-\alpha(T)}}{\kappa r_0^a}R_0^\delta B_0(1-\delta^2)  c
\,.
\end{equation*}
As a consequence, one  has 
\begin{equation*}
 C=2 \text{e}^{-\alpha(r_H)} E_K\,.
\end{equation*}
This is an universal form of the action growth (valid also for ``dirty'' black holes with $\alpha(r)\neq 0$), where
the Kodama-Hayward energy,
\begin{equation*}
E_H=\text{e}^{-\alpha(r_H)} E_K\,,
\end{equation*}
appears. 
\end{frame}

\begin{frame}
{\bf Black hole phenomenology}

In GR the Hamiltonian on a three-surface $\Sigma$ bounded by a sphere which is a part of a horizon has the form,
\begin{equation*}
{\cal H}=\mathrm{bulk\;\,term}-\frac{1}{8\pi G_N}\int_{\cal M}(\kappa-
16\pi h^{-1/2}P^{ij}N_i\xi_j)dA +\mathrm{terms\;\,at\;\,infinity}\,,
\end{equation*}
where $\kappa$ is the surface gravity, $N_{i}$ the shift, $\xi_{j}$ the normal to ${\cal M}$ within the three-surface. On shell the bulk term vanishes, the momentum term also vanishes in a 
static geometry, while the term at infinity is absent if $\Sigma$ is internal to the horizon.
Identifying the temperature $T=\kappa/2\pi$ and the entropy $S=A/4G_N$, the action grow bound is 
$\dot{I}=TS_{BH}\leq 2E$, or, by taking the
derivatives,
\begin{equation*}
\dot{S}_{BH}\leq -2P/T\,,
\end{equation*}
where $P$ is the power emitted by the BH. The generalized second law leads to, 
\begin{equation*}
0\leq \dot{S}_{BH}+\dot{S}_{rad} \rightarrow
2P/T\leq \dot{S}_{rad}\,,
\end{equation*}
where $\dot{S}_{rad}$ is the entropy carried away by the Hawking radiation.
\end{frame}
 
\begin{frame}
 If one accepts  Pendry's universal bound [Pendry, 1983] on the entropy flow out of a thermal source radiating in vacuum (like the black hole),
\begin{equation*}
\dot{S}_{rad}\leq \left(\frac{\pi P}{3}\right)^{1/2}\,,
\end{equation*}
one gets for the power the limit,
\begin{equation*}
P\leq \frac{\pi T^2}{12}\,.
\end{equation*}
All the bounds are saturated by the Schwarzschild AdS black hole, and we have seen that the result can be generalized to the framework of $F(R)$-gravity. 
\end{frame}

\begin{frame}
{\bf Deser-Sarioglu-Tekin black holes}
The action is given by,
\begin{equation*}
I = \frac{1}{16\pi G_N} \int_{\mathcal M}\,d^4x\,\sqrt{-g} 
\left(R -2\Lambda + \sqrt{3}\sigma\,\sqrt{W}\right) \,,
\end{equation*}
where $W=C_{\mu\nu\xi\sigma}C^{\mu\nu\xi\sigma}$ is the square of the Weyl tensor. After integration by parts on SSS metric we get
\begin{eqnarray*}
 I&=&\frac{V_k}{16\pi G_N}\int dt\int dr
 \text{e}^{\alpha(r)}\left(
 -2\Lambda r^2+2k(1-\epsilon\sigma)+
 2B(r)(1-\epsilon\sigma)
 \right.\nonumber\\&&\left.+
 2r \frac{d B(r)}{d r}(1-4\epsilon\sigma)
 +2rB(r)\frac{d\alpha(r)}{d r}(2-5\epsilon\sigma)
 \right)+BT\,,
\end{eqnarray*}
where the boundary term is ($\epsilon=\pm 1$ such that $\sqrt{W}=|\sqrt{W}|$),
\begin{eqnarray*}
&&BT=\nonumber\\
&&
\hspace{-1.5cm}-\frac{V_k}{16\pi G_N}\int dt\int dr\frac{d}{dr}\left[
\text{e}^{\alpha(r)}r^2\left(\frac{d B(r)}{d r}+2B(r)\frac{d\alpha(r)}{d r}+\frac{4B(r)}{r}\right)\left(
1-\epsilon\sigma
\right)
\right]\,.
\end{eqnarray*}
\end{frame}
 
 \begin{frame}
The field equations are at the second order and admit the solution [Deser et al. 2008],
\begin{equation*}
ds^2=\left[\frac{r}{r_0}\right]^{\frac{6\epsilon\sigma}{\epsilon\sigma-1}}B(r)dt^2+\frac{dr^2}{B(r)}+r^2d\Omega^2_k\,,
\end{equation*}
\begin{equation*}
B(r)= k\,\frac{(1-\epsilon\sigma)}{(1-4\epsilon\sigma)} - c r^{-\frac{1-4\epsilon\sigma}{1-\epsilon\sigma}}
 -\Lambda\,\frac{r^2}{3(1-2\epsilon\sigma)}\,,\quad\sigma\neq \pm1\,,\pm\frac{1}{4}\,.
\end{equation*}
Here, $c$ is an integration constant. Given the black hole solution with $B(r_H)=0\,, B'(r_H)>0$, we can derive the First Law as,
\begin{equation*}
T_K d S_W=\frac{V_k\text{e}^{\alpha(r_H)}}{\kappa}\left[
(1-\epsilon\sigma)k-\Lambda r_H^2\right]\equiv d E_K\,,
\end{equation*}
where $T_K$ is the BH Killing temperature and the Wald entropy can be computed as,
\begin{equation*}
 S_W=(4\pi )\frac{V_k r_H^2}{2\kappa}\left(1-\epsilon\sigma\right)\,.
 \end{equation*}
 \end{frame}

\begin{frame}
 The growth action of the black hole can be computed in an analogue way of the $F(R)$-case by starting from the bulk action and the result is
\begin{equation*}
C=\text{e}^{-\alpha(r_H)}E_K\frac{(2-5\epsilon\sigma)}{(1-\epsilon\sigma)}\leq 2\text{e}^{-\alpha(r_H)}E_K\,.
\end{equation*}
In this case of modified gravity model, the action growth does not coincide with the double of the Kodama energy, but is still proportional
to it and, more importantly, bounded by twice its value as long as $0<\epsilon\sigma$, which is in accord with the general complexity bound as usually stated.  
In the contrary case with $\epsilon\sigma<0$ the bound is violated. 
As a check, when $\sigma $ goes to zero, one gets the result of General Relativity.
\end{frame}

\begin{frame}
 
\end{frame}

 
 
\end{document}
