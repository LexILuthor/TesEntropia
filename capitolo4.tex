\chapter{Conclusioni}
\label{cha:conclusioni}
Se ci si concentra solo sulla riduzione dell'errore commesso si può essere portati a sottovalutare la portata del teorema di Shannon, infatti per ridurre l'errore si potrebbe pensare semplicemente di inviare più volte il simbolo che deve essere inviato, in questo modo essendo che per ogni simbolo inviato la probabilità d'errore è $p$ avremmo che perché il sistema registri un errore nella ricezione bisognerebbe avere che vengano trasmessi errati almeno $\frac{n}{2}$ simboli e questo evento verrebbe modellizzato attraverso una variabie casuale binomiale che al crescere di $n$ farebbe tendere la probabilità d'errore a zero. Questo procedimento però al crescere di $n$ ridurrebbe la velocità di trasmissione mandando anch'essa a zero. La forza del teorema fondamentale di Shannon sta proprio in quest'osservazione e cioè ci garantisce l'esistenza di un codice che pur mandando a zero l'errore commesso mantiene la velocità di trasmissione di informazione arbitrariamente vicina alla capacità del canale. Purtroppo questo teorema però non è del tipo costruttivo, non ci fornisce cioè un metodo per la creazione di tale codice.\\
È stato dimostrato che l'inverso non è possibile cioè non si può avere una probabilità d'errore arbitrariamente piccola se si trasmette ad una capacità superiore a quella del canale.\\
Come accennato nella discussione del teorema ricordiamo che  è stato dimostrato che è possibile controllare non solo la probabilità media di errore, ma anche quella massima ($\max_{1\leq i \leq M} \mathbb{P}(E|x_i)$).\\







