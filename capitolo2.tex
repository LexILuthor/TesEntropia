\chapter{Entropia per Variabili Casuali Assolutamente Continue}
\vspace{15pt}

In questo capitolo estenderemo la definizione di entropia data per il caso di una variabile aleatoria discreta al caso in cui la nostra variabile casuale $X$ sia assolutamente continua.

\begin{defi}
Data una variabile casuale $X$, chiamiamo \textbf{funzione di distribuzione di X} l'applicazione $F_X: \mathbb{R} \to \mathbb{R}$ data da:
$$F_X(t):= \mathbb{P}(X \in (-\infty,t])$$
\end{defi}
\begin{defi}
Una funzione di distribuzione $F$ è detta \textbf{assolutamente continua} se esiste una funzione $f \in L^1(\mathbb{R})$, $f\geq 0$ e $\int_{\mathbb{R}} f(u)du=1$ tale che:
\begin{equation}\label{eq:assConti}
F(t)=\int_{-\infty }^t f(u)du, \  t\in \mathbb{R}
\end{equation}
dove l'integrale è definito nel senso di Lebesgue. Tale $f$ verrà detta \textbf{funzione di densità}\\
Una variabile casuale che ha funzione di distribuzione della forma \ref{eq:assConti} è detta \textbf{variabile casuale assolutamente continua}
\end{defi}
Per le proprietà degli elementi appena definiti si veda \cite{Mazzucchi}

\vspace{15pt}


\section{Entropia nel caso Continuo}
\label{sec:EntropiaContinuo}
\vspace{10pt}

\begin{defi}
Sia $X$ una variabile casuale con immagine $(a,b)$ e funzione di densità $f$. $H(X)$ detta \textbf{entropia di X} dove:
$$H(X)=-\int_a^b \log(f(x))f(x)dx= \mathbb{E}\bigg[ \log \bigg( \frac{1}{f(X)} \bigg) \bigg]$$
\end{defi}
Anche qui per convenzione $\log$ sarà il logaritmo in base 2, inoltre formalmente considereremo $\log(0)=0$. \\
Purtroppo la proprietà di essere misura di incertezza, valida nel caso discreto \ref{teo:misuraIncertezza}, non è più valida con questa definizione.\\
Questo deriva dal fatto che, mentre nel caso discreto l'argomento del logaritmo è sempre compreso tra 0 e 1, nel caso continuo la funzione di densità, argomento del logaritmo, può assumere valori su tutto $\mathbb{R}$.\\
Per un esempio si calcoli l'entropia associata alla variabile casuale uniforme, ricordando che la su  funzione di densità è $f(x)=\frac{1}{b-a}$ si ottiene:
\[
\begin{split}
H(X)& = \int_a^b \log \bigg( \frac{1}{b-a} \bigg) \frac{1}{b-a} dx \\
& =\log(b-a)
\end{split}
\]
che sarà negativa se $0 < b-a < 1$.\\
L'entropia per le variabili casuali non potrà quindi giocare un ruolo così importante come quello giocato per variabili casuali discrete. Esistono tuttavia alcuni teoremi degni di nota.\\
Prima di introdurli però calcoliamo l'entropia di $X \backsim N(\mu, \sigma^2)$.
\[
\begin{split}
H(X)& = - \frac{1}{\sigma (2 \pi)^{\frac{1}{2}}} \int_{-\infty}^{\infty} \exp \bigg(- \frac{1}{2} \bigg( \frac{x-\mu}{\sigma} \bigg)^2 \bigg) \log \bigg( \frac{1}{\sigma (2\pi)^{\frac{1}{2}}} \exp \bigg( - \frac{1}{2} \bigg( \frac{x-\mu}{ \sigma} \bigg)^2 \bigg)  \bigg) dx  \\ 
&=\frac{1}{\sigma(2 \pi )^{1/2}} \log (\sigma(2 \pi )^{1/2}) \int_{-\infty}^{\infty} \exp ( -y^2 ) \sigma 2^{1/2} dy - \frac{1}{\sigma (2 \pi)^{1/2}} \frac{1}{\ln (2)}   \int_{-\infty}^{\infty} -y^2 \exp (-y^2) \sigma 2^{1/2} dy \\
& = \log (\sigma (2 \pi)^{\frac{1}{2}})- \frac{1}{\ln(2) \pi^{\frac{1}{2}}}\int_{-\infty}^{\infty} e^{-y^2}(-y^2) dy
\end{split}
\]
dove nella seconda riga abbiamo usato la sostituzione $y=\frac{x-\mu}{2^{1/2}\sigma}$.\\
Il calcolo diretto dell'integrale al secondo membro risulta:
$$\int_{-\infty}^{\infty} e^{-y^2}(-y^2) dy= - \frac{\pi^{\frac{1}{2}}}{2}$$
e quindi sostituendo nell'equazione sopra:
\begin{equation}
\begin{split}
H(X)&=\log (\sigma (2\pi)^{\frac{1}{2}})+\frac{1}{\ln(2) \pi^{\frac{1}{2}}}\frac{\pi^{\frac{1}{2}}}{2}\\
&=\log (\sigma (2\pi)^{\frac{1}{2}})+\frac{1}{2 \ln(2)} \\
&=\log (\sigma (2\pi)^{\frac{1}{2}})+ \log (e^{\frac{1}{2}}) \\
&=\log ( \sigma (2\pi e)^{\frac{1}{2}})
\end{split}
\end{equation}
D'ora in avanti indicheremo $\log(\sigma(2\pi e)^{\frac{1}{2}})$ con $H_N(\sigma)$. Questo evidenzia il fatto che varianza ed entropia sono due concetti molto legati.

\begin{teo}\label{teo:GibbsContinuo}(\textit{Disuguaglianza di Gibbs nel caso continuo})
Siano $f,g$ due funzioni di densità allora vale
\begin{equation}
-\int_{-\infty}^{\infty} \log(f(x))f(x)dx \leq - \int_{-\infty}^{\infty} \log(g(x))f(x)dx
\end{equation}
dove l'uguaglianza è valida solo se $g(x)=f(x)$.
\end{teo}
\begin{proof}
Dato che $\log_b(a)=\frac{\ln (a)}{\ln (b)}$ possiamo limitarci al caso in cui abbiamo $\ln(x)$, il quale ha la proprietà di essere sempre maggiore di $x-1$ e uguale solo nel caso $x=1$. Quindi
\[
\begin{split}
- \int_{-\infty}^{\infty} \log(g(x))f(x) +  \int_{-\infty}^{\infty} \log(f(x))f(x)dx & = - \int_{-\infty}^{\infty} \bigg[\log(g(x))-\log(f(x))\bigg] f(x)dx \\
& =-\int_{-\infty}^{\infty} \log \bigg(\frac{g(x)}{f(x)}\bigg)f(x)dx\\
& \geq  -\int_{-\infty}^{\infty} \bigg(\frac{g(x)}{f(x)} -1\bigg)f(x)dx\\
&= -\int_{-\infty}^{\infty} g(x)dx +\int_{-\infty}^{\infty} f(x)dx=0
\end{split}
\]
Dove l'ultima uguaglianza si ottiene ricordando che $f,g$ sono funzioni di densità e quindi $\int_{-\infty}^{\infty} f(x)dx=1, \ \int_{-\infty}^{\infty} g(x)dx=1$.\\
La disuguaglianza nel caso precedente diventa un uguaglianza solo se $\frac{g(x)}{f(x)}=1$ cioè solo se $g(x)=f(x)$.\\
Notiamo che, nel caso in cui $f(x)$ si annulla possiamo utilizzare un procedimento analogo a quello usato nella dimostrazione del punto 2 del teorema \ref{teo:6.2} di nel caso in cui $p_j=0$.\\
\end{proof}


\begin{teo}
Sia $X$ una variabile casuale assolutamente continua con immagine $\mathbb{R}$ di media $\mu$, varianza $\sigma^2$ e funzione di densità $f$ allora
$$H(X) \leq H_N(\sigma)$$
con l'uguaglianza se e solo se $X\backsim N(\mu,\sigma^2)$
\end{teo}
\begin{proof}
Dalla disuguaglianza di Gibbs \ref{teo:GibbsContinuo} appena dimostrata otteniamo che, per ogni funzione di densità $g$:
$$H(X)\leq - \int_{-\infty}^\infty \log (g(x))f(x)dx$$
con l'uguaglianza solo se $f(x)=g(x)$.
Come $g$ prendiamo $\g$ cioè la funzione di densità di una Normale $N\backsim N(\mu, \sigma^2)$
\[
\begin{split}
- \int_{-\infty}^\infty \log (g(x))f(x)dx &= - \int_{-\infty}^\infty \log \bigg( \g\bigg)f(x)dx\\
&= \log(\sigma(2\pi)^{1/2})- \frac{\log(e)}{2\sigma^2}\int_{-\infty}^{\infty} (y-\mu)^2f(y)dy \\
&= \frac{1}{2} \log(2\pi \sigma^2)+ \frac{ \log(e)}{2\sigma^2} Var(X)\\
&=\frac{1}{2} \log(2\pi \sigma^2)+ \frac{ \log(e)}{2}\\
&=\frac{1}{2} \log(2\pi e \sigma^2)=H_N(\sigma)
\end{split}
\]
Dove c'è la disuguaglianza avere l'uguaglianza dobbiamo avere $f=g$ e quindi $X\backsim N(\mu,\sigma^2)$
\end{proof}









